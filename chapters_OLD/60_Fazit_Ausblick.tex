\chapter{Fazit und Ausblick}


\section{Zusammenfassung der wichtigsten Erkenntnisse}
Im Praktikum konnte ich meine Kenntnisse über KI und deren Integration in Softwareprojekte umfassend erweitern. Das
arbeiten mit Kollegen im Team so wie das Entwickeln agiler Software mithilfe von Scrum waren eine erfrischende und
spannende Abwechslung. Ich habe für mich viel dazugelernt, auch abseits des Schreibens von Quellcode habe ich gelernt
wie man wichtige Entscheidungen und Erkenntnisse festhält und dokumentiert, wie man neue Technologien einsetzt und auch
wie man Software plant und veröffentlicht.


\section{Einordnung der erlernten Inhalte und Fähigkeiten im Kontext meines Studiums}
Meine Programmierkenntnisse aus Programmieren 1 und 2 konnte ich beim Programmieren mit Java und Javascript einsetzten und beim
Programmieren mit Python erweitern. Bei der Anpassung der Datenbank im TenderSniffer waren die gelernten Inhalte über
relationale Datenbanken aus dem Modul Datenbanken sehr hilfreich. Am meisten profitiert habe ich neben dem Wissen aus
Programmieren 1 und 2 von Software-Technik, da mich hier besonders viele Themen abseits von Quellcode schreiben auf
meine Arbeit als Softwareentwickler vorbereitet haben. Dazu gehören Versionskontrollen wie Git, Softwareentwicklungsmodelle wie Scrum und
Kanban, aber auch DevOps Themen wie CI/CD Pipelines und das Testen von Code.


\section{Weiterer Werdegang}
Meine Erfahrungen im Praxissemester haben mich darin bestätigt meine berufliche Zukunft im Bereich der
Softwareentwicklung zu gestalten. Ich habe viel Freude daran Software zu entwickeln und im Team an Lösungen zu arbeiten. 
Eine berufliche Zukunft bei iteratec kann ich mir sehr gut vorstellen, da die Arbeitsbedingungen durch einen hohen Grad
an Betreuung und Flexibilität, sowie die unfassbar freundlichen und sympathischen Kollegen für eine leistungsorientierte
aber zugleich sehr harmonische Arbeitsatmosphäre sorgen in der ich mich sehr wohl gefühlt habe.
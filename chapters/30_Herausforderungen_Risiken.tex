\chapter{Herausforderungen und Risiken}

In diesem Kapitel werden Beispiele für die Herausforderungen und Risiken im Zusammenhang mit
der Nutzung von KI im Bereich des Datenschutzes erläutert. 

% Dabei liegt der Fokus auf den potenziellen Datenschutzbeeinträchtigungen durch KI.

% \section{Datenlecks}

% KI-Systeme können anfällig für Datenlecks sein, bei denen sensible Informationen
% unbefugt offengelegt werden. Ein Datenleck kann durch Schwachstellen in der
% IT-Infrastruktur oder durch menschliches Versagen entstehen. Solche Vorfälle
% können schwerwiegende Folgen haben, wie Identitätsdiebstahl oder finanziellen
% Betrug. Um das Risiko von Datenlecks zu minimieren, müssen Unternehmen
% sicherstellen, dass sie über robuste Sicherheitsmaßnahmen verfügen, wie z.B.
% regelmäßige Sicherheitsüberprüfungen und den Einsatz von
% Verschlüsselungstechnologien.

\section{Profiling und Diskriminierung}

Algorithmen können personenbezogene Daten analysieren und daraus Profile
erstellen, die zu Diskriminierung führen können. Dies geschieht häufig unbemerkt
und kann tiefgreifende Auswirkungen auf die betroffenen Personen haben.
Beispielsweise könnten Bewerber aufgrund von algorithmischen Entscheidungen von
Bewerbungsverfahren ausgeschlossen werden oder Verbraucher könnten unfairen
Kreditentscheidungen ausgesetzt sein. Unternehmen müssen sicherstellen, dass
ihre KI-Systeme fair und transparent sind und regelmäßig auf
Diskriminierungspotenzial überprüft werden.

\section{Unerwünschte Datenverarbeitung}

KI-Systeme könnten personenbezogene Daten für Zwecke nutzen, die nicht im
Einklang mit den ursprünglichen Erhebungszwecken stehen. Dies kann zu einer
unerwünschten Verarbeitung von Daten führen, die gegen Datenschutzbestimmungen
verstößt. Beispielsweise könnten Daten, die für die Verbesserung eines Dienstes
erhoben wurden, ohne Zustimmung der betroffenen Personen für Marketingzwecke
verwendet werden. Unternehmen müssen klare Richtlinien für die Datennutzung
festlegen und sicherstellen, dass die Verwendung von Daten stets im Einklang mit
den ursprünglichen Erhebungszwecken und den Einwilligungen der Nutzer steht.

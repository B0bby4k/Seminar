\chapter{Tätigkeitsbereich und Aufgabenstellung}
In diesem Kapitel werde ich genauer auf meinen Arbeitsbereich und meine Aufgaben eingehen.
In den darauf folgenden Kapiteln werde ich ein Projekt (TenderInsights) ausführlich beschreiben, und ein anderes nur
grob umreißen (TenderSniffer). An beiden Projekten habe ich ungefähr 3 Monate gearbeitet.

\section{Arbeitsbereich und Ausstattung}
Sämtliche Arbeitsplätze sind sogenannte Flex Arbeitsplätze. Das heißt, dass man sich morgens einen freien Schreibtisch
sucht und dort seine Arbeit beginnt. Es gibt vereinzelt Bereiche wie das SLAB, wo sich bestimmte Gruppen zum arbeiten
treffen aber auch dort hat niemand einen festen Arbeitsplatz. Jeder dieser Arbeitsplätze verfügt über einen
höhenverstellbaren Schreibtisch und 2 Monitore inklusive Dockingstation für den Laptop. Jedem Mitarbeiter ist es
gestattet seine Arbeit von zuhause aus durchzuführen, solange man mindestens einen Tag in der Woche vor Ort im Büro
arbeitet. Ein Großteil der Kommunikation mit Kollegen erfolgt daher über Videokonferenzen oder Chats mithilfe der
Software Microsoft Teams. Für Meetings stehen zahlreiche Meeting Räume zur Verfügung, welche man sich jederzeit
reservieren kann. Für die Arbeit erhält man als Student einen Dell Laptop mit Windows-Betriebssystem. 


\section{Hauptaufgaben}
Zu meinen Hauptaufgaben gehört insbesondere das Entwickeln von Software, beziehungsweise das Schreiben von Quellcode.
Eigenständiges recherchieren und Problemlösung gehören dabei fest zum Arbeitsalltag. Entwickelt wird iterativ nach einer
abgewandelten Kanban-Variation. So werden Tickets für einzelne Features oder gefundene Bugs in Jira im Backlog angelegt.
Die Entwicklung erfolgt in mehreren Sprints, bei denen Aufgaben aus dem Backlog entnommen und in die "`New"' -Spalte
geschoben werden. Sobald mit der Arbeit an einer Aufgabe begonnen wurde, schiebt man das Ticket in die "`In
Progress"'-Spalte, bei erledigten Aufgaben kommen diese in die "`Review"'-Spalte, bei Problemen oder Unklarheit in die
"`Waiting"'-Spalte bis eine Entscheidung gefunden wurde wie man fortfahren will. Bei den ein- oder zwei-wöchentlichen
Meetings werden die "`New"'-Tickets unter den Entwicklern aufgeteilt und mögliche Probleme oder Unklarheiten besprochen.
Aufgaben, welche sich in "`Review"' befinden kommen in die "`Done"'-Spalte falls alles wie gewünscht funktioniert und der
Product Owner zufrieden ist. Da es keine festen WIP-Limits (Work in Progress) gibt kann nicht von reinem Kanban
gesprochen werden, allerdings ist es eher unüblich dass ein Entwickler mehr als 2 Aufgaben parallel bearbeitet.


\section{Verwendete Software und Grundlagen}
In diesem Kapitel werden alle Software-Werkzeuge, welche während der gesamten Dauer meines Praktikums Projektübergreifend
verwendet werden, erläutert.


\subsection{Werkzeuge}

\subsubsection{Visual Studio Code}
Visual Studio Code (VSCode) ist ein leistungsfähiger und anpassbarer Quelltext-Editor von Microsoft, der eine breite
Palette von Programmiersprachen unterstützt und zahlreiche Erweiterungen für Entwickler bietet. 

\subsubsection{GitLab}
GitLab ist eine auf Git basierende webbasierte DevOps-Plattform, welche für die Versionskontrolle und CI/CD bei der
Softwareentwicklung verwendet wird. Für die Testdateien und Dokumentation verwenden wir zusätzlich LFS (Large File
Storage) um größere Dateien im Repository abzulegen.

\subsubsection{Microsoft Teams}
Ein Kommunikationstool von Microsoft, welches genutzt wird um sich mit Kollegen auszutauschen, Dateien zu teilen und
Besprechungen zu planen so wie durchzuführen.

\subsubsection{Jira}
Ein Projektmanagement-Werkzeug von Atlassian, das speziell für Softwareentwicklungsteams konzipiert ist und Funktionen
für die Verfolgung von Aufgaben, Bugs und agile Prozesse bietet. Es kam insbesondere beim Verwalten des Backlogs und
beim Bearbeiten von Tickets zum Einsatz.

\subsubsection{Miro}
Eine digitale, kollaborative Whiteboard-Plattform, die es unserem Team ermöglicht, visuell zu brainstormen, zu planen
und Projekte in Echtzeit zu gestalten.

\subsection{Fachlich}

\subsubsection{PMO und Akquise}
Das Projektmanagement Office strukturiert und definiert Projektmanagement im Unternehmen. Es arbeitet eng mit der
Projektakquise zusammen, welche nach neuen potentiellen Projekten bei Kunden und öffentlichen Ausschreibungen suchen.
\chapter{Einleitung}
In der heutigen digitalisierten Welt spielen künstliche Intelligenzen (KI) eine
immer wichtigere Rolle in vielen Bereichen unseres Lebens. Von personalisierten
Empfehlungen in Online-Shops über automatisierte Kundenbetreuung bis hin zu
intelligenten Assistenzsystemen im Gesundheitswesen, die Anwendungen von KI sind
vielfältig und ihre Potenziale enorm. Doch mit dem rasanten Fortschritt dieser
Technologien wachsen auch die Bedenken bezüglich des Datenschutzes. Datenschutz
bezeichnet den Schutz von Daten, insbesondere personenbezogener Informationen,
vor Missbrauch und unbefugter Verarbeitung. Die Herausforderung besteht darin,
die Vorteile der KI zu nutzen, während gleichzeitig die Privatsphäre der
Menschen geschützt wird. Die Integration von KI-Systemen in so viele Aspekte des
täglichen Lebens führt zu einer massiven Sammlung und Analyse von Daten, oft in
einer Weise, die die Grenzen traditioneller Datenschutzmaßnahmen testet oder
sogar überschreitet. KI kann Muster und Zusammenhänge in Daten erkennen, die für
das menschliche Auge unsichtbar sind, was sowohl Chancen als auch Risiken birgt.
Einerseits kann dies zur Optimierung von Prozessen, zur Verbesserung von
Dienstleistungen und zur Förderung wissenschaftlicher und medizinischer
Forschung beitragen. Andererseits kann dies auch zu einer unerwünschten oder
sogar illegalen Überwachung und Profilbildung führen, wenn die gesammelten Daten
missbraucht werden. Der vorliegende Bericht zielt darauf ab, ein tiefes
Verständnis dafür zu schaffen, wie KI-Systeme datenschutzrelevante
Herausforderungen darstellen und welche gesetzlichen sowie technischen Maßnahmen
erforderlich sind, um die Privatsphäre der Menschen in einer zunehmend von KI
dominierten Welt zu schützen. Er beleuchtet die aktuellen Datenschutzgesetze,
die speziell für den Umgang mit KI entwickelt wurden, und untersucht, inwieweit
diese ausreichend sind, um den einzigartigen Herausforderungen, die KI stellt,
gerecht zu werden. 

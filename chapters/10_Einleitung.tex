\chapter{Einleitung}
In diesem Bericht schildere ich die Aufgaben und Projekte, welche ich in meinem Praxissemester im Wintersemester 23/24
bei dem Unternehmen iteratec GmbH durchgeführt und begleitet habe. Dabei gehe ich auf die eingesetzten Technologien ein
und Teile meine gesammelten Erfahrungen und ordne diese inhaltlich im Rahmen meines Studiums der Wirtschaftsinformatik
ein.

\section{Unternehmen iteratec GmbH}

Iteratec ist ein Unternehmen, das sich auf digitale Produktinnovation, Software- und Architekturentwicklung sowie
digitale Infrastrukturen spezialisiert hat. Als Partner für die digitale Transformation unterstützt es Unternehmen und
öffentliche Organisationen bei der Digitalisierung ihrer Prozesse, Produkte und Dienstleistungen. Dies umfasst die
Entwicklung von Innovationen, die technische Umsetzung individueller Softwarelösungen, deren Betrieb und
Weiterentwicklung sowie Schulungen in agilen Methoden. Das Unternehmen entwickelt mobile Applikationen,
Cloud-Architekturen und bietet Application Performance Management an. Es nutzt auch Technologien wie Blockchain, AI und
IoT für die Entwicklung neuer Geschäftsmodelle. Zu den Kunden gehören mittelständische Unternehmen, DAX-Konzerne und
Organisationen aus verschiedenen Branchen. 1996 in München gegründet, hat iteratec Standorte in Deutschland, Österreich
und Polen und beschäftigt rund 500 Mitarbeiter und 120 Studierende. Seit 2019 sind viele Mitarbeiter über eine
Genossenschaft am Unternehmen beteiligt.

\section{Name und Geschichte}
Das Unternehmen wurde 1996 gegründet und existiert bereits seit über 25 Jahren. Während zu dieser Zeit das
Wasserfallmodell in der Softwareentwicklung weit verbreitet war hat sich iteratec schon früh auf agile Methoden und
Werte spezialisiert. Erkennbar ist das insbesondere an dem Namen, welcher sich aus den Begriffen iterativ und
Technologie zusammensetzt. Heute hat iteratec bereits über 1000 Softwareprojekte erfolgreich abgeschlossen.

\section{Unternehmensstruktur}
Standortübergreifend hat iteratec 4 Hauptgeschäftsführer. An jedem Standort gibt es jeweils einen
Geschäftsstellenleiter. Jeder Mitarbeiter hat eine persönliche Führungskraft (pFK), welche für die persönliche
Entwicklung vor allem im, aber auch außerhalb des Unternehmens zuständig ist. Jedes Projekt hat einen Product Owner
(PO), welcher die Interessen des Kunden im Projekt vertritt, und einen Projektleiter, welcher dafür zuständig ist die
Fristen und eingesetzten Ressourcen zu managen. Alle Mitarbeiter begegnen sich Standort- und Positionsübergreifend auf
Augenhöhe, was sich unter anderem dadurch zeigt, dass sich alle duzen. Die bisherigen Alleingesellschafter der iteratec
GmbH haben im Jahr 2018 entschieden, das Unternehmen weder innerhalb ihrer Familie zu vererben noch dieses an externe zu
verkaufen. Daraus entstand die iteratec nurdemtean eG, eine Genossenschaft bei der sich die Mitarbeiter am Unternehmen
beteiligen, mit dem Ziel, die Unternehmenskultur zu halten.

\section{Studierende bei iteratec (SLAB)}
Die Abteilung für studierende, das SLAB (Studentenlabor), ist Standort übergreifend und schließt sowohl alle
Studierenden als auch deren Betreuer mit ein. Halb-jährlich werden neue SLAB Leiter gewählt, deren Aufgabe es ist, alle
anfallenden Aufgaben an studierende zu delegieren und im engen Austausch mit dem SLAB-Verantwortlichen zu stehen. Eine
Aufgabe ist das Bereitstellen einer Startbegleitung für neue Studierende. Andere Aufgaben sind z.B. das Organisieren von
Workshops und Events.

\section{Meetings und Veranstaltungen}
Bei iteratec gibt es abseits der regulären Projektmeetings viele weitere Veranstaltungen an denen man teilnehmen kann.
Im folgenden Kapitel erläutere ich die wichtigsten.

\subsection{SLAB-Meeting}
Das SLAB-Meeting findet zwei Mal pro Jahr statt und beschränkt sich auf den Standort. Hier wird die SLAB-Leitung
gewählt, welche immer aus zwei Studenten besteht. Es wird retrospektiv analysiert und bewertet, wie gut die Prozesse rund um
die Studentenbetreuung funktionieren. Es werden Verbesserungsvorschläge gesammelt und hinsichtlich Nutzen und Durchführbarkeit
bewertet. Anschließend werden entsprechende Maßnahmen an verantwortliche verteilt, deren Aufgabe es ist sicherzustellen, dass diese
durchgeführt werden. Das eigenverantwortliche Organisieren der Studenten hat viele Vorteile, darunter das Sammeln von
praktischen Erfahrungen in Führung, Management und Teamarbeit, als auch aktiv Einfluss auf die
Unternehmenspolitik und -kultur auszuüben.

\subsection{GS-Meeting}
Das Geschäftsstellen-Meeting für die Geschäftsstelle Stuttgart, an dem alle Mitarbeiter des Standorts eingeladen sind,
findet etwa ein Mal pro Monat statt. Hier werden aktuelle Themen besprochen und Mitarbeiter können Ihre Projekte
vorstellen. Die Geschäftsführung verkündet während des Meetings wichtige Neuigkeiten, anstehende Events und präsentiert
die aktuellen Geschäftszahlen. Neue Mitarbeiter und studierende werden begrüßt und ausscheidende Mitarbeiter
verabschiedet. Zum Ende können Kollegen für die Wall of Fame in einer der drei Leitwerte von iteratec (Permanent
Exzellent, Konstruktiv unabhängig und Inspirierend mutig) nominieren und so besondere Leistungen und gute zusammenarbeit
würdigen.

\subsection{BUILD23}
Bei der BUILD handelt es sich um eine Messe, welche ausschließlich von iteratec Mitarbeitern besucht werden darf. Sie
geht über zwei Tage und fand dieses Jahr bei der Geschäftsstelle in München statt. Im Vorfeld kann jeder Mitarbeiter,
Student oder jedes Team sich ein Thema überlegen und als Vorschlag einreichen. Nachdem die Themen freigegeben wurden
plant man einen entsprechenden Messestand über drei bis sechs Zeitslots, in denen man sein Thema, Projekt oder Workshop
präsentieren kann. Hierzu gibt es an jedem Stand Ausrüstung in Form eines TVs, eines Mikrofons und mehreren Receiver mit
Kopfhörern. Ziel der Build ist es das erlangte Wissen Projekt- und Standortübergreifend auszutauschen und sich mit
Kollegen zu vernetzen oder Interessensgemeinschaften zu bilden. Im Anschluss an die Build findet die Weihnachtsfeier
statt.

%
% Generelle Hinweise:
% - Werfen Sie auch einen Blick in die Word-Vorlage, falls dort Hinweise sind, die hier nicht enthalten sind.
%-  (Dieses Dokument ist für einseitigen Druck formatiert; wenn zweiseitig gedruckt werden soll, müssen die Seitenzahlen
%   und Header entsprechend angepasst werden.) 
%-  Auf Abbildungen / Tabellen wird möglichst im Text vor der Abbildung verwiesen. Ist in Latex manchmal schwierig
%-  Abbildungen sollten nach Möglichkeit so groß dargestellt sein, dass auch die Texte gut lesbar sind; es sei denn
%   diese sind völlig bedeutungslos und nur die Struktur oder das Gesamtbild sind von Bedeutung.
%-  Bei farbigen Abbildungen sollte sichergestellt werden, dass diese auch in Schwarz-Weiß gut erkennbar sind.
%-  Tabellen sollten zweckmäßig und übersichtlich sein: Vermeidung unnötiger Linien, Farbgebung nur, wenn sie eine
%   Bedeutung hat oder der Übersichtlichkeit dient.
%-  Zitiert wird typischerweise nach APA. Alternativen sind aber möglich (mit den Betreuer:innen klären). 


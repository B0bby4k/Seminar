\chapter{Tätigkeitsbereich und Aufgabenstellung}
Im folgenden Kapitel werde ich genauer auf meinen Arbeitsbereich und meine Aufgaben eingehen.

\section{Arbeitsbereich und Ausstattung}
Sämtliche Arbeitsplätze sind sogenannte Flexarbeitsplätze. Das heißt, dass man sich morgens einen freien Schreibtisch sucht und dort seine Arbeit beginnt. Es gibt vereinzelt Bereiche wie das SLAB, wo sich bestimmte Gruppen zum arbeiten treffen aber auch dort hat niemand einen festen Arbeitsplatz. Jeder dieser Arbeitsplätze verfügt über einen höhenverstellbaren Schreibtisch und 2 Monitore inklusive Dockingstation für den Laptop. Jedem Mitarbeiter ist es gestattet seine Arbeit von Zuhause aus durchzuführen, solange man mindestens einen Tag in der Woche vor Ort im Büro arbeitet. Ein Großteil der Kommunikation mit Kollegen erfolgt daher über Videokonferenzen oder Chats mithilfe der Software Microsoft Teams. Für Meetings stehen zahlreiche Meeting räume zur Verfügung, welche man sich jederzeit reservieren kann.
Für die Arbeit erhält man als Student einen Dell Laptop mit Windows-Betriebssystem. 

\section{Hauptaufgaben}
Zu meinen Hauptaufgaben gehörte insbesondere das entwickeln von Software beziehungsweise das schreiben von Quellcode. Eigenständiges recherchieren und Problemlösung gehören dabei fest zum Arbeitsalltag.


\section{Technologien und Tools}
Für das schreiben von Code verwende ich als Entwicklungsumgebung die IDE Visual Studio Code. 
\chapter{Lösungsansätze}

\section{Technische Maßnahmen}
Ein zentraler Lösungsansatz zur Bewältigung der Datenschutzprobleme im
Zusammenhang mit Künstlicher Intelligenz (KI) besteht in der Implementierung
technischer Maßnahmen. Eine wichtige Maßnahme ist die Verschlüsselung von Daten,
um sensible Informationen vor unbefugtem Zugriff zu schützen. Unternehmen
sollten moderne Verschlüsselungstechnologien einsetzen und regelmäßig
aktualisieren.

Anonymisierungs- und Pseudonymisierungstechniken sind ebenfalls entscheidend.
Anonymisierte Daten unterliegen nicht den strengen Vorgaben der
Datenschutz-Grundverordnung (DSGVO), und pseudonymisierte Daten können das
Datenschutzrisiko verringern \cite{conrad_2017}.

Regelmäßige Sicherheitsüberprüfungen und Penetrationstests sind notwendig, um
potenzielle Schwachstellen in IT-Systemen zu identifizieren und zu beheben.
Diese Maßnahmen minimieren das Risiko von Datenlecks und unbefugtem Zugriff.

\section{Organisatorische Maßnahmen}
Neben technischen Maßnahmen spielen auch organisatorische Maßnahmen eine
wichtige Rolle beim Schutz personenbezogener Daten. Dazu gehört die Entwicklung
klarer Datenschutzrichtlinien und -verfahren sowie die Schulung von Mitarbeitern
im Umgang mit sensiblen Daten.

Datenschutz-Folgenabschätzungen (DSFA) sind gemäß Art. 35 DSGVO erforderlich,
bevor neue Technologien eingeführt werden. Diese Bewertungen helfen, potenzielle
Datenschutzrisiken frühzeitig zu erkennen und geeignete Maßnahmen zu ergreifen
\cite{conrad_2017}.

Die Ernennung eines Datenschutzbeauftragten, der die Einhaltung der
Datenschutzvorschriften überwacht, ist ebenfalls wichtig. Der
Datenschutzbeauftragte sollte Schulungen und Workshops für Mitarbeiter
organisieren, um das Bewusstsein für Datenschutzthemen zu schärfen.

Klare Richtlinien für die Nutzung und Verarbeitung personenbezogener Daten sind
entscheidend. Diese Richtlinien sollten die Zwecke der Datenerhebung und
-verarbeitung genau definieren und sicherstellen, dass personenbezogene Daten
nicht für andere Zwecke verwendet werden, als ursprünglich vorgesehen.
Mechanismen zur Wahrung der Rechte der betroffenen Personen auf Auskunft,
Berichtigung und Löschung ihrer Daten sollten implementiert werden
\cite{conrad_2017}.

Durch die Kombination technischer und organisatorischer Maßnahmen können
Unternehmen die Datenschutzrisiken im Zusammenhang mit der Nutzung von KI
erheblich reduzieren und sicherstellen, dass ihre KI-Systeme im Einklang mit den
gesetzlichen Datenschutzanforderungen stehen.


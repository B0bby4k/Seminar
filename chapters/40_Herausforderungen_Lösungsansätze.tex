\chapter{Herausforderungen und Lösungsansätze}

% In diesem Kapitel werden Beispiele für die Herausforderungen und Risiken im
% Zusammenhang mit der Nutzung von KI im Bereich des Datenschutzes erläutert.
% Anschließend werden mögliche Lösungsansätze für die angeführten
% Herausforderungen erörtert.

\section{Herausforderungen und Risiken}

% \subsection{Datenlecks}

% KI-Systeme können anfällig für Datenlecks sein, bei denen sensible Informationen
% unbefugt offengelegt werden. Ein Datenleck kann durch Schwachstellen in der
% IT-Infrastruktur oder durch menschliches Versagen entstehen. Solche Vorfälle
% können schwerwiegende Folgen haben, wie Identitätsdiebstahl oder finanziellen
% Betrug. Um das Risiko von Datenlecks zu minimieren, müssen Unternehmen
% sicherstellen, dass sie über robuste Sicherheitsmaßnahmen verfügen, wie z.B.
% regelmäßige Sicherheitsüberprüfungen und den Einsatz von
% Verschlüsselungstechnologien.

\subsection{Profiling und Diskriminierung}

Algorithmen können personenbezogene Daten analysieren und daraus Profile
erstellen, die zu Diskriminierung führen können. Dies geschieht häufig unbemerkt
und kann tiefgreifende Auswirkungen auf die betroffenen Personen haben.
Beispielsweise könnten Bewerber aufgrund von algorithmischen Entscheidungen von
Bewerbungsverfahren ausgeschlossen werden oder Verbraucher könnten unfairen
Kreditentscheidungen ausgesetzt sein. 

% Unternehmen müssen sicherstellen, dass
% ihre KI-Systeme fair und transparent sind und regelmäßig auf
% Diskriminierungspotenzial überprüft werden.

\subsection{Unerwünschte Datenverarbeitung}

KI-Systeme könnten personenbezogene Daten für Zwecke nutzen, die nicht im
Einklang mit den ursprünglichen Erhebungszwecken stehen. Dies kann zu einer
unerwünschten Verarbeitung von Daten führen, die gegen Datenschutzbestimmungen
verstößt. Beispielsweise könnten Daten, die für die Verbesserung eines Dienstes
erhoben wurden, ohne Zustimmung der betroffenen Personen für Marketingzwecke
verwendet werden.


\section{Mögliche Lösungsansätze}

\subsection{Technische Maßnahmen}
% Ein zentraler Lösungsansatz zur Bewältigung der Datenschutzprobleme im
% Zusammenhang mit Künstlicher Intelligenz (KI) besteht in der Implementierung
% technischer Maßnahmen. Eine wichtige Maßnahme ist die Verschlüsselung von Daten,
% um sensible Informationen vor unbefugtem Zugriff zu schützen. Unternehmen
% sollten moderne Verschlüsselungstechnologien einsetzen und regelmäßig
% aktualisieren.

% Anonymisierungs- und Pseudonymisierungstechniken sind ebenfalls entscheidend.
% Anonymisierte Daten unterliegen nicht den strengen Vorgaben der
% Datenschutz-Grundverordnung (DSGVO), und pseudonymisierte Daten können das
% Datenschutzrisiko verringern \cite{conrad_2017}.

Eine technische Schutzmaßnahm um Datenschutzprobleme im Zusammenhang mit KI zu
vermeiden ist die Verschlüsselung von Daten und sensiblen Information mithilfe
von modernen Verschlüsselungstechnologien. Zusätzlich kann mithilfe von
Anonymisierungs- und Pseudonymisierungstechniken das Datenschutzrisiko
verringert werden, da diese nicht den strengen Vorgaben der DSGVO unterliegen.
Es sollten auch Mechanismen zur Wahrung der Rechte der betroffenen Personen auf
Auskunft, Berichtigung und Löschung ihrer Daten implementiert werden.

\cite{conrad_2017}

\subsection{Organisatorische Maßnahmen}

Datenschutz-Folgenabschätzungen sind gemäß Art. 35 DSGVO erforderlich,
bevor neue Technologien eingeführt werden. Diese Bewertungen helfen, potenzielle
Datenschutzrisiken frühzeitig zu erkennen und geeignete Maßnahmen zu ergreifen.

Auch organisatorische Maßnahmen spielen eine wichtige Rolle beim Schutz
personenbezogener Daten. Die Ernennung eines Datenschutzbeauftragten, der die
Einhaltung der Datenschutzvorschriften überwacht, und klare
Datenschutzrichtlinien und -verfahren entwickelt ist ebenfalls wichtig. Das
Anbieten von Schulungen und Workshops für die Mitarbeiter unterstützen dabei
zusätzlich, da hierdurch das Bewusstsein für Datenschutzthemen gestärkt wird.

Art. 35 der DSGVO schreibt vor, vor der Einführung neuer Technologien sogenannte
Datenschutz-Folgenabschätzungen durchzuführen. Sie helfen, Datenschutzrisiken
frühzeitig zu erkennen und entsprechende Maßnahmen zu ergreifen.

\cite{conrad_2017}
\chapter{Herausforderungen und Lösungsansätze}

\section{Herausforderungen und Risiken}

\subsection{Datenlecks}

Es besteht die Gefahr, dass sensible Informationen in falsche Hände geraten und
für betrügerische Zwecke verwendet werden. Dies kann zum Beispiel durch
Cyberangriffe, menschliches Versagen oder unzureichende Datensicherung
eintreffen.

\cite{digitalesinstitut2024}

\subsection{Unerwünschte Datenverarbeitung}

KI-Systeme könnten personenbezogene Daten für Zwecke nutzen, die nicht im
Einklang mit den ursprünglichen Erhebungszwecken stehen. Dies kann zu einer
unerwünschten Verarbeitung von Daten führen, die gegen Datenschutzbestimmungen
verstößt. Beispielsweise könnten Daten, die für die Verbesserung eines Dienstes
erhoben wurden, ohne Zustimmung der betroffenen Personen für Marketingzwecke
verwendet werden. Für die Verarbeitung von in KI Modelle eingegebene
personenbezogenen Daten liegt in der Regel keine Rechtsgrundlage vor. Grund ist,
dass betroffene oftmals vorher nicht explizit in die Verarbeitung eingewilligt
haben. 

\cite{digitalesinstitut2024}


\subsection{Profiling und Diskriminierung}

KI-Systeme, vorwiegend Lernsysteme, sind abhängig von den erfassten Daten. Wenn
Datengrundlagen unzureichend sind, können diese Systeme Ergebnisse präsentieren,
die sich als diskriminierend erweisen. Dies geschieht häufig unbemerkt und kann
tiefgreifende Auswirkungen auf die betroffenen Personen haben. Unternehmen
müssen sicherstellen, dass ihre KI-Systeme fair und transparent sind und
regelmäßig auf Diskriminierungspotenzial überprüft werden.

\cite{keyed2024}

\subsection{Rechtliche Rahmenbedingungen}



Die Grundsätze der Datenverarbeitung sehen gemäß Art. 83 Abs. 5 DSGVO ein
Bußgeld bis zu 20 Mio. Euro oder 4 Prozent des weltweit erzielten Jahresumsatzes
des vorangegangenen Geschäftsjahres, je nachdem welcher Betrag größer ist.
Verantwortliche müssen zudem die Umsetzung der Grundsätze für die Verarbeitung
personenbezogener Daten gewährleisten, indem technisch-organisatorische
Maßnahmen gemäß Art. 24 DSGVO ergriffen werden.
\cite{keyed2024}

\section{Mögliche Lösungsansätze}

\subsection{Technische Maßnahmen}

Eine technische Schutzmaßnahme um Datenschutzprobleme im Zusammenhang mit KI zu
vermeiden, ist die Verschlüsselung von Daten und sensiblen Information mithilfe
von modernen Verschlüsselungstechnologien. Zusätzlich kann mithilfe von
Anonymisierungs- und Pseudonymisierungstechniken das Datenschutzrisiko
verringert werden, da diese nicht den strengen Vorgaben der DSGVO unterliegen.
Das verarbeiten der Daten kann auch auf lokale Geräte eingeschränkt werden, wenn
man entsprechende Modelle entwickelt. Es sollten auch Mechanismen zur Wahrung
der Rechte der betroffenen Personen auf Auskunft, Berichtigung und Löschung
ihrer Daten implementiert werden.

\cite{conrad_2017}
\cite{mindverse2024}

\subsection{Organisatorische Maßnahmen}

Auch organisatorische Maßnahmen spielen eine wichtige Rolle beim Schutz
personenbezogener Daten. Die Ernennung eines Datenschutzbeauftragten, der die
Einhaltung der Datenschutzvorschriften überwacht, und klare
Datenschutzrichtlinien und -verfahren entwickelt, ist ebenfalls wichtig. Das
Anbieten von Schulungen und Workshops für die Mitarbeiter unterstützt dabei
zusätzlich, da hierdurch das Bewusstsein für Datenschutzthemen gestärkt wird.

Art. 35 der DSGVO schreibt vor, vor der Einführung neuer Technologien sogenannte
Datenschutz-Folgenabschätzungen durchzuführen. Sie helfen, Datenschutzrisiken
frühzeitig zu erkennen und entsprechende Maßnahmen zu ergreifen.

\cite{conrad_2017}
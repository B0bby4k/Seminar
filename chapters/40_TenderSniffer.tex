\chapter{TenderSniffer}
Dieses Kapitel nur kurz.

\section{Kurze Projektbeschreibung}
TenderSniffer ist ein Projekt, welches aus mehreren Applikationen besteht und zum Ziel hat verschiedene
Ausschreibungsportale nach neuen Ausschreibungen zu durchsuchen und die gefundenen Ausschreibungen visuell strukturiert
darzustellen. Teilanwendungen hierbei sind der TenderCrawler, welcher die Ausschreibungen von den einzelnen Plattformen
zieht und in die Datenbank speichert. Der TenderWeb ist ein Graphisches User Interface (GUI), welches die
Ausschreibungen in der Datenbank darstellt und Benutzereingaben abspeichert. Mithilfe von verschiedenen Schaltflächen
kann die Ausschreibung entweder an die Akquiseabteilung weitergeleitet werden oder für unpassend deklarieren.

\section{Verwendete Software und Grundlagen}
In diesem Kapitel werden alle SW-Werkzeuge und Technologien beschrieben, welche im Entwicklungsumfeld des TenderSniffers
verwendet werden oder für ein Verständnis zuträglich sind.

\subsection{Werkzeuge}

\subsubsection{IntelliJ IDEA}
Integrierte Entwicklungsumgebung des Softwareunternehmens JetBrains. Wird für die Entwicklung mit Java und Kotlin
eingesetzt. 

\subsubsection{PostgreSQL}
Freies Datenbankmanagementsystem, welches seit 1997 von einer Open-Source-Community weiterentwickelt wird. Es wird oft
mit Postgres abgekürzt.

\subsubsection{SonarQube}
Plattform für statische Analyse und Bewertung von Quelltext. Die Ergebnisse der Analyse werden über eine Website dargestellt.

\subsubsection{Lint}
Werkzeug, welches den Quellcode auf Fehler, unübliche Muster, nicht eingehaltene Stilrichtlinien und potenzielle Bugs
überprüfen, um eine bessere Codequalität und -konsistenz zu gewährleisten.


\subsection{Technologien}

\subsubsection{Java}
Java ist eine objektorientierte höhere Programmiersprache welche 2010 von Oracle übernommen wurde.
Sie findet neben Computerapplikationen auch Einsatz bei Apps für Smartphones, Tablets und Spielekonsolen.

\subsubsection{JavaScript}
JavaScript ist eine vielseitige und weit verbreitete Programmiersprache, die hauptsächlich für die Entwicklung von
interaktiven Webseiten und Webanwendungen eingesetzt wird.

\subsubsection{Spring Boot}
Spring Boot ist ein Java-Framework zur Vereinfachung der Entwicklung und Bereitstellung von Spring-basierten
Anwendungen, das auf Konvention statt Konfiguration setzt und zahlreiche Funktionen für einen schnellen Projektstart
bietet.

\subsubsection{Vue.js}
Vue.js ist ein JavaScript-Framework, das für den Aufbau interaktiver Web-Oberflächen und Single-Page-Applikationen verwendet wird.

\subsubsection{Jest}
Jest ist ein JavaScript-Test-Framework, das für seine einfache Konfiguration und effiziente Leistung bei der Entwicklung
von Webanwendungen bekannt ist. Jest ist ein JavaScript-Test-Framework, das häufig für das Testen von React- und
Vue.js-Anwendungen eingesetzt wird, wobei es Funktionen wie einfache Konfiguration und integrierte Mocking-Unterstützung
bietet.

\section{Meine Rolle und Aufgaben im Projekt}
Ich bin in der Rolle eines Softwareentwickler zum Projekt hinzugestoßen, es wird agil mithilfe von Jiraboards und
Sprints entwickelt. Ein Sprint geht über vier Wochen, da ein großteil der Projektbeteiligten Werkstudenten sind und
entsprechend nur ein bis zwei Tage pro Woche arbeiten. Die Endnutzer (PMO- und Akquiseabteilungen) erstellen
eigenständig neue tickets und beschreiben darin aus Benutzersicht welche Funktionen, Änderungen und Fehler gewünscht
sind, bzw. beseitigt werden sollen. Neben den täglichen Austauschterminen, den sogenannten "Dailies" gibt es jeden
Freitag einen größeren Besprechungstermin, die "Weeklies". Am ende eines jeden Sprints findet ein "Review", bevor der
nächste Sprint begonnen wird.

\subsection{Kommentare ausblenden}
Meine Einstiegsaufgabe war es in der Kommentarsektion einen Button im Frontend des TenderWeb zu programmieren, mit dem
der Kommentarbereich eingeklappt werden kann sobald dieser mehr als 3 Kommentare enthält. Hierfür habe ich die
vorhandene commentBox.vue um den besagten button erweitert und mithilfe von javascript die Logik hinter dem button
implementiert. Nach mehreren Schleifen mit dem Team und den Nutzern habe ich für die finale Version tests mithilfe von
jest geschrieben. Im Anschluss habe ich den Merge Request erstellt und nachdem die DevOps Pipeline erfolgreich
durchgelaufen ist und die Funktion auf dem Testbuild einwandfrei funktioniert hat wurde das Feature auf der
Produktionsumgebung ausgeliefert und das Ticket abgeschlossen.

\subsection{Auftraggeberdetails erfassen}
Die zweite und umfangreichste Aufgabe war es, die Daten der Auftraggeber in einer eigenen Box anzuzeigen und editierbar
zu gestalten, so dass Änderungen konsistent sind. Hierfür habe ich das Datenbankschema um eine client-tabelle erweitert
und im Backend entsprechende domain-, repository-, service- und Controllerklassen geschrieben, mit denen über HTTP
Requests die API angesprochen werden kann und Auftaggeber sowohl abgerufen als auch gespeichert werden können. Im
Frontend habe ich den store, welcher als zentraler Ort zur Speicherung und Verwaltung des Anwendungszustands, was eine
konsistente Datenverwaltung und -verteilung über die gesamte Anwendung hinweg ermöglicht, um entsprechende Methoden
erweitert. Anschließend habe ich eine clientBox mit vue erstellt, welche diese Daten aus dem store anzeigt und bei
Änderungen mithilfe von store Mutationen und Aktionen die neuen Daten über eine POST Request an das Backend übermittelt
und somit konsistent in der Datenbank abgespeichert werden. 



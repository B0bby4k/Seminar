\chapter{TenderSniffer}
Dieses Kapitel nur kurz.

\section{Kurze Projektbeschreibung}
TenderSniffer ist ein Projekt, welches aus mehreren Applikationen besteht und zum Ziel hat verschiedene
Ausschreibungsportale nach neuen Ausschreibungen zu durchsuchen und die gefundenen Ausschreibungen visuell strukturiert
darzustellen. Teilanwendungen hierbei sind der TenderCrawler, welcher die Ausschreibungen von den einzelnen Plattformen
zieht und in die Datenbank speichert. Der TenderWeb ist ein Graphisches User Interface (GUI), welches die
Ausschreibungen in der Datenbank darstellt und Benutzereingaben abspeichert. Mithilfe von verschiedenen Schaltflächen
kann die Ausschreibung entweder an die Akquiseabteilung weitergeleitet werden oder für unpassend deklarieren.

\section{Verwendete Software und Grundlagen}
In diesem Kapitel werden alle SW-Werkzeuge, Technologien und Begriffe beschrieben, welche im Entwicklungsumfeld des TenderSniffers
verwendet werden oder für ein Verständnis zuträglich sind.

\subsection{Werkzeuge}

\subsubsection{IntelliJ IDEA}
Integrierte Entwicklungsumgebung des Softwareunternehmens JetBrains. Wird für die Entwicklung mit Java und Kotlin
eingesetzt. 

\subsubsection{PostgreSQL}
Freies Datenbankmanagementsystem, welches seit 1997 von einer Open-Source-Community weiterentwickelt wird. Es wird oft
mit Postgres abgekürzt.

\subsubsection{SonarQube}
Plattform für statische Analyse und Bewertung von Quelltext. Die Ergebnisse der Analyse werden über eine Website dargestellt.

\subsection{Technologien}

\subsubsection{Java}
Java ist eine objektorientierte höhere Programmiersprache welche 2010 von Oracle übernommen wurde.
Sie findet neben Computerapplikationen auch Einsatz bei Apps für Smartphones, Tablets und Spielekonsolen.

\subsection{Wichtige Begriffe}

\begin{table}[H]
    \centering
    \caption{Wichtige Begriffe}
    \label{tab:technologien}
    \begin{tabular}{|>{\centering\arraybackslash}m{4cm}|p{10cm}|} % Anpassen der Breitenangaben nach Bedarf
    \hline
    \textbf{Name} & \textbf{Beschreibung} \\ \hline
    Name & Beschreibung1 \\ \hline
    \end{tabular}
\end{table}

\section{Meine Rolle und Aufgaben im Projekt}
Ich bin in der Rolle eines Softwareentwickler zum Projekt hinzugestoßen.


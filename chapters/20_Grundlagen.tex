\chapter{Grundlagen der KI und des Datenschutzes}

\section{Künstliche Intelligenz (KI)}

Künstliche Intelligenz (KI) bezeichnet Technologien, die es Computern
ermöglichen, Aufgaben welche normalerweise menschliche Intelligenz benötigen,
durchzuführen. Wichtige Bestandteile sind dabei das Lernen, Problemlösen und
Verstehen von Sprache. Ein zentrales Element ist das maschinelle Lernen
bei dem riesige Datenmengen mithilfe von Algorithmen Muster formen und
Vorhersagen treffen. Mit der Verbreitung von Large Language Models wie
GPT-3.5 von OpenAI hat die Zugänglichkeit und Nutzung von KI stark zugenommen. 
\cite{haerting2024}

\section{Datenschutz}

Datenschutz schützt personenbezogene Daten vor Missbrauch und unbefugtem
Zugriff. In der EU regelt die Datenschutz-Grundverordnung (DSGVO) diesen Schutz.
Personenbezogene Daten umfassen sämtliche Informationen, die eine Person
identifizieren, wie Namen, Adressen und pseudonymisierte Daten wie
Kundennummern. Die DSGVO erlaubt die Verarbeitung solcher Daten nur auf
rechtmäßiger Grundlage, wie durch eine Einwilligung oder das berechtigte
Interesse des Datenverarbeiters.

\cite{haerting2024}

\chapter{Grundlagen der KI und des Datenschutzes}
In diesem Kapitel werden die grundlegenden Konzepte der Künstlichen Intelligenz
(KI) und des Datenschutzes erläutert. Dabei liegt der Fokus auf den Technologien
hinter KI und den datenschutzrechtlichen Anforderungen gemäß DSGVO.

\section{Künstliche Intelligenz (KI)}
Künstliche Intelligenz (KI) bezeichnet Technologien, die es Computern
ermöglichen, Aufgaben zu übernehmen, die normalerweise menschliche Intelligenz
erfordern. Dazu gehören das Lernen, Problemlösen und Verstehen von Sprache. Ein
prominentes Beispiel ist die natürliche Sprachverarbeitung (NLP), die es
Systemen wie Chatbots ermöglicht, menschliche Anfragen zu verstehen und zu
beantworten. Ein zentrales Element moderner KI ist das maschinelle Lernen, bei
dem Algorithmen aus großen Datenmengen Muster erkennen und Vorhersagen treffen
können. Neuronale Netzwerke, inspiriert vom menschlichen Gehirn, spielen dabei
eine wichtige Rolle und ermöglichen komplexe Analysen und Entscheidungen.

Mit der Verbreitung von Large Language Models (LLMs) wie GPT-3.5 von OpenAI hat
die Zugänglichkeit und Nutzung von KI stark zugenommen. Diese Modelle
verarbeiten immense Datenmengen und können beeindruckend präzise Antworten
generieren, was jedoch auch datenschutzrechtliche Fragen aufwirft.

\section{Datenschutz}
Datenschutz ist der Schutz personenbezogener Daten vor Missbrauch und
unerlaubtem Zugriff. In der Europäischen Union wird dieser Schutz durch die
Datenschutz-Grundverordnung (DSGVO) gewährleistet. Personenbezogene Daten sind
laut Art. 4 Nr. 1 DSGVO alle Informationen, die sich auf eine identifizierte
oder identifizierbare Person beziehen. Dies umfasst Namen, Adressen, aber auch
pseudonyme Daten wie Kundennummern, sofern sie mit zusätzlichen Informationen
verknüpft werden können.

Die DSGVO stellt sicher, dass personenbezogene Daten nur auf rechtmäßiger
Grundlage verarbeitet werden dürfen. Dies schließt die Einwilligung der
betroffenen Person (Art. 6 Abs. 1 lit. a DSGVO) und das berechtigte Interesse
des Datenverarbeiters (Art. 6 Abs. 1 lit. f DSGVO) ein. 

Unternehmen müssen sicherstellen, dass die Datenverarbeitung im Einklang mit den
Datenschutzbestimmungen steht. Dazu gehört auch die Verpflichtung,
Betroffenenrechte zu wahren, wie das Recht auf Auskunft, Berichtigung und
Löschung der Daten. Die Einhaltung dieser Regelungen ist entscheidend, um hohe
Strafen zu vermeiden und das Vertrauen der Nutzer zu gewinnen.
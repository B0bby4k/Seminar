\chapter{Ausblick auf die Zukunft von KI und Datenschutz}

Der Einsatz von Künstlicher Intelligenz wird in den kommenden Jahren
voraussichtlich weiter zunehmen und tiefgreifende Veränderungen in vielen
Lebensbereichen bewirken. Während KI-Technologien weiterentwickelt werden,
müssen die Datenschutzregelungen entsprechend angepasst werden, um neuen
Herausforderungen gerecht zu werden.

Es ist zu erwarten, dass zukünftig strengere gesetzliche Vorgaben und
fortschrittlichere technische Lösungen erforderlich werden, um den Datenschutz
zu gewährleisten. Ein wichtiger Aspekt wird die Entwicklung von KI-Systemen
sein, die von Anfang an datenschutzfreundlich konzipiert sind (Privacy by Design
und Privacy by Default), so wie es zum Beispiel Apple mit Apple Intelligence auf
der Hauseigenen Messe WDC24 angekündigt hat. Hier werden personengbezogene Daten
nur mit einer lokal betriebenen KI verarbeitet. Sollte die KI mit einer Anfrage
überfordert sein, soll zwar zur Unterstützung chat-GPT hinzugezogen werden,
jedoch sollen dabei keine bzw nur anonymisierte oder verschlüsselte Daten
übermittelt werden. Dadurch soll gewährleistet werden, dass diese Informationen
das Gerät des Nutzers nicht verlassen.

Insgesamt zeigt sich, dass der verantwortungsbewusste Einsatz von Künstlicher
Intelligenz nur möglich ist, wenn Datenschutz und ethische Überlegungen fest in
den Entwicklungs- und Implementierungsprozessen verankert sind. Unternehmen,
Regierungen und die Gesellschaft als Ganzes müssen zusammenarbeiten, um eine
Balance zwischen technologischem Fortschritt und dem Schutz der individuellen
Privatsphäre zu finden.

\cite{mindverse2024}
